%%%%%%%%%%%%%%%%%%%%%%%%%%%%%%%%%%%%%%%%%%%%%%%%%%%%%%%%%%%%%%%%%%%%%%%%
% Preamble
%%%%%%%%%%%%%%%%%%%%%%%%%%%%%%%%%%%%%%%%%%%%%%%%%%%%%%%%%%%%%%%%%%%%%%%%
\documentclass[11pt]{article}
%
% Packages and other includes
% Pagination
\usepackage[letterpaper, margin=1.25in]{geometry}
%
% Fonts
\usepackage[T1]{fontenc} % best for Western European languages
\usepackage{lmodern} % Latin Modern instead of CM
\usepackage{textcomp} % required to get special symbols
%
% Math
\usepackage{amsmath, amssymb}
\usepackage{braket}
%
% Graphics, floats, tables
\usepackage{graphicx, xcolor, float, array}
\usepackage{subcaption}
%
% Hyperlinks
\usepackage{hyperref}
%
% Bibliography
\usepackage[style=numeric, sorting=none, backend=biber]{biblatex}
\addbibresource{references.bib}
%
% Revision (see Makefile)
%\input{revision.tex}
%
% Definitions and settings
% Paragraph indent and spacing
\setlength{\parskip}{0.4\baselineskip}
\setlength{\parindent}{0in}
%
% Math mode version of "r" column type (requires array package)
\newcolumntype{R}{>{$}r<{$}}
%
%comments
\newcommand{\brian}[1]{{\color{orange} #1}}
% Title, authors, date
\title{\textbf{Is Using TPSS and TPSSh for RPA better for noncovalent
    calculations?}}
\author{Thanh Huynh and Brian Nguyen}
\date{11/27/2020 -- 01/10/2021 }
%
%
%%%%%%%%%%%%%%%%%%%%%%%%%%%%%%%%%%%%%%%%%%%%%%%%%%%%%%%%%%%%%%%%%%%%%%%%
% Main document
%%%%%%%%%%%%%%%%%%%%%%%%%%%%%%%%%%%%%%%%%%%%%%%%%%%%%%%%%%%%%%%%%%%%%%%%
%

\begin{document}

\maketitle

\section{Introduction}

\brian{Mention big picture -> lissoclimide project; follow-up to the
  previous report}

A potent cytotoxin, called chlorolissoclimide, was discovered
in the sea squirt and could potentially be useful in cancer therapeutics
or other pharmaceuticals, such as medication for COVID-19. This chemical
induces cell death by obstructing production of proteins. An interesting
ribosome-drug interaction was revealed to form a halogen-pi bond however,
characteristics of the bond are still unknown. Halogen bonds are formed
when a halogen interacts with other atoms within a molecule.Researching
this specific interaction could help with creating new cancer therapeutic
or improving medication.

Following up the previous report, random phase approximation (RPA) was
chosen to be used with different functionals. The TPSS and TPSSh
functionals may describe the halogen-pi bond more accurately than PBE.
To investigate which functional performs better, the X40 test set serves
as a benchmark for TPSS, TPSSh, and PBE calculations.  


\section{Methods}

Based on the previous report, the 3-4 extrapolation of the RIRPA
correlation energy was a good balance between efficiency and accuracy.
The X40 Test Set was used to benchmark the method's accuracy for 
noncovalent interactions involving halogens. The test set contains 40
complexes with despersion, induction, dipole-dipole, stack, hydrogen
bond, halogen bond, and halogen-pi bond interations.

To conserve time, a set of eight complexes from the X40 test set were
chosen based on error from the previous report to study basis set
convergences. Calculations were computed with basis sets def2-QZVP,
cc-pVTZ, and cc-pVQZ. The RPA energies were computed based on converged
orbitals from TPSS and TPSSh and resolution of identity (RI) was included
to improve efficiency. The Hartree Exact Exchange (HXX) total energy and
RIRPA correlation energy were recorded from the supermolecule and
monomers of each complex to compute interaction energies. The interaction
energies from the selected complexes were extrapolated using the
supramolecular approach. Due to the basis set superposition error (BSSE),
counterpoise corrections (CP) were applied to the RIRPA correlation
energy as well.

\section{Results}

\begin{figure}[H]
  \centering
  \begin{subfigure}{\textwidth}
    \center
    \includegraphics[scale=0.35]{def2-QZVP_1.png}
    \label{fig:def2-QZVP_1}
  \end{subfigure}
  \begin{subfigure}{\textwidth}
    \center
    \includegraphics[scale=0.35]{def2-QZVP_2.png}
    \label{fig:def2-QZVP_2}
  \end{subfigure}
  \caption{The binding energy errors (kcal/mol) for X40 test set computed
    for RPA(PBE) methods with basis functions def2-QZVP, cc-pVTZ, and
    cc-pVQZ. Negative sign indicates overbinding.}
  \label{fig:def2-QZVP Error}
\end{figure}

For most complexes, the method using the def2-QZVP basis function yields
an error greater than the 3-4 extrapolated, 50 percent CP
extrapolated, and cc-pVQZ. 

\begin{table}[hbpt]
  \caption{here is caption}
  \centering
  \begin{tabular}{c|cccc}
       & col1 & col2 & col3 & col4 \\
    \hline\hline
    ME & blah & blah & blah & blah \\
    MAE & blah & blah & blah & blah
  \end{tabular}
  \label{tab:stuff}
\end{table}

mean error
def2-QZVP: 0.6030545998
3-4 extrapolate: 0.3895447736
50 percent 3-4 extrapolate: 0.5980591124
cc-pVTZ: 0.8334508798
cc-pVQZ: 0.5768176622


\begin{figure}[H]
  \centering
  \begin{subfigure}{.5\textwidth}
    \centering
    \includegraphics[scale=0.3]{tpss-1.png}
    \caption{RPA(TPSS)}
    \label{fig:tpss_1}
  \end{subfigure}%
  \begin{subfigure}{.5\textwidth}
    \centering
    \includegraphics[scale=0.3]{tpssh-1.png}
    \caption{RPA(TPSSh)}
    \label{fig:tpssh_1}
  \end{subfigure}
  \caption{Basis sets convergence plot for RPA(TPSS) and RPA(TPSSh) is
    presented for complex 1. Dunning's basis sets were used for all atoms
    and 1/X$^3$, where X is the cardinal number, was used for
    extrapolation to form linear lines. Complex 1 contains dispersion
    interaction with fluorine.}
  \label{fig:complex_1}
\end{figure}

The convergence plot for RPA(TPSS) and RPA(TPSSh) both don't appear to be
converging since the extrapolated lines are nearly parallel. The plot for
RPA(TPSSh) also approaches a value that is more negative than RPA(TPSS).


\begin{figure}[H]
  \centering
  \begin{subfigure}{.5\textwidth}
    \centering
    \includegraphics[scale=0.3]{tpss-8.png}
    \caption{RPA(TPSS)}
    \label{fig:tpss_8}
  \end{subfigure}%
  \begin{subfigure}{.5\textwidth}
    \centering
    \includegraphics[scale=0.3]{tpssh-8.png}
    \caption{RPA(TPSSh)}
    \label{fig:tpssh_8}
  \end{subfigure}
  \caption{Basis sets convergence plot for RPA(TPSS) and RPA(TPSSh) is
    presented for complex 8. Dunning's basis sets were used for all atoms
    and 1/X$^3$, where X is the cardinal number, was used for extrapolation
    to form linear lines. Complex 8 contains induction interactions with
    chlorine.}
  \label{fig:complex_8}
\end{figure}

The convergence plot for RPA(TPSS) all intersect before approaching 0
whereas the plot for RPA(TPSSh) appears to converge. 


\begin{figure}[H]
  \centering
  \begin{subfigure}{.5\textwidth}
    \centering
    \includegraphics[scale=0.3]{tpss-11.png}
    \caption{RPA(TPSS)}
    \label{fig:tpss11}
  \end{subfigure}%
  \begin{subfigure}{.5\textwidth}
    \centering
    \includegraphics[scale=0.3]{tpssh-11.png}
    \caption{RPA(TPSSh)}
    \label{fig:tpssh_11}
  \end{subfigure}
  \caption{Basis sets convergence plot for RPA(TPSS) and RPA(TPSSh) is
    presented for complex 11. Dunning's basis sets were used for all
    atoms and 1/X3, where X is the cardinal number, was used for
    extrapolation to form linear lines. Complex 11 contains stack
    interactions with fluorine.}
  \label{fig:complex_11}
\end{figure}

Both the RPA(TPSS) and RPA(TPSSh) plots for complex 11 are converging
to one value as they approach 0. The interaction energy for RPA(TPSS)
appears to be negative while the energy for RPA(TPSSh) is positive.


\begin{figure}[H]
  \centering
  \begin{subfigure}{.5\textwidth}
    \centering
    \includegraphics[scale=0.3]{tpss-24.png}
    \caption{RPA(TPSS)}
    \label{fig:tpss_24}
  \end{subfigure}%
  \begin{subfigure}{.5\textwidth}
    \centering
    \includegraphics[scale=0.3]{tpssh-24.png}
    \caption{RPA(TPSSh)}
    \label{fig:tpssh_24}
  \end{subfigure}
  \caption{Basis sets convergence plot for RPA(TPSS) and RPA(TPSSh) is
    presented for complex 24. Dunning's basis sets were used for all
    atoms and 1/X3, where X is the cardinal number, was used for
    extrapolation to form linear lines. Complex 24 contains halogen 
    bonding with iodine.}
  \label{fig:complex_24}
\end{figure}

For complex 24, both RPA(TPSS) and RPA(TPSSh) do not converge at all.
The lines for both plots are parallel to each other as they approach 0.


\begin{figure}[H]
  \centering
  \begin{subfigure}{.5\textwidth}
    \centering
    \includegraphics[scale=0.3]{tpss-27.png}
    \caption{RPA(TPSS)}
    \label{fig:tpss_27}
  \end{subfigure}%
  \begin{subfigure}{.5\textwidth}
    \centering
    \includegraphics[scale=0.3]{tpssh-27.png}
    \caption{RPA(TPSSh)}
    \label{fig:tpssh_27}
  \end{subfigure}
  \caption{Basis sets convergence plot for RPA(TPSS) and RPA(TPSSh) is
    presented for complex 27. Dunning's basis sets were used for all
    atoms and 1/X3, where X is the cardinal number, was used for
    extrapolation to form linear lines. Complex 27 contains halogen-pi
    bonding with bromine.}
  \label{fig:complex_27}
\end{figure}

The RPA(TPSS) and RPA(TPSSh) convergence plots look similar and both don't
converge to one point. The extrapolated lines from cc-pVTZ to cc-pVQZ
appear nearly parallel with each other.

\begin{figure}[H]
  \centering
  \begin{subfigure}{.5\textwidth}
    \centering
    \includegraphics[scale=0.3]{tpss-30.png}
    \caption{RPA(TPSS)}
    \label{fig:tpss_30}
  \end{subfigure}%
  \begin{subfigure}{.5\textwidth}
    \centering
    \includegraphics[scale=0.3]{tpssh-30.png}
    \caption{RPA(TPSSh)}
    \label{fig:tpssh_30}
  \end{subfigure}
  \caption{Basis sets convergence plot for RPA(TPSS) and RPA(TPSSh) is
    presented for complex 30. Dunning's basis sets were used for all
    atoms and 1/X3, where X is the cardinal number, was used for
    extrapolation to form linear lines. Complex 30 contains halogen-pi
    bonding with iodine.}
  \label{fig:complex_30}
\end{figure}

The convergence plot for RPA(TPSS) and RPA(TPSSh) both do not converge
as the extrapolated lines approach 0. The lines for RPA, 50 percent CP,
and 100 percent CP appear to be parallel.

\begin{figure}[H]
  \centering
  \begin{subfigure}{.5\textwidth}
    \centering
    \includegraphics[scale=0.3]{tpss-38.png}
    \caption{RPA(TPSS)}
    \label{fig:tpss_38}
  \end{subfigure}%
  \begin{subfigure}{.5\textwidth}
    \centering
    \includegraphics[scale=0.3]{tpssh-38.png}
    \caption{RPA(TPSSh)}
    \label{fig:tpssh_38}
  \end{subfigure}
  \caption{Basis sets convergence plot for RPA(TPSS) and RPA(TPSSh) is
    presented for complex 38. Dunning's basis sets were used for all
    atoms and 1/X3, where X is the cardinal number, was used for
    extrapolation to form linear lines. Complex 38 contains hydrogen
    bonding with chlorine.}
  \label{fig:complex_38}
\end{figure}

The convergence plots both don't converge as the lines approach 0. The
extrapolated line from cc-pVTZ to cc-pVQZ for RPA and 50 percent CP
appear to be parallel for the RPA(TPSS) and RPA(TPSSh) plots.

\brian{yada yada see Fig. \ref{fig:<name>}}




\section{Conclusions}

- def2-QZVP is a good balance between accuracy and efficiency ?

Using TPSS and TPSSh functionals for RPA do not describe the different
interactions better than PBE.


This section should answer the following questions:
What is the significance of your results for the question, problem, or
hypothesis posed in the introduction? What, if any, are the
implications in the broader context of the research area laid out in the
second paragraph of the introduction? 

A second paragraph should discuss any conclusions for your future work. 
Is there a need to change your approach or your priorities? What remains
to be done to reach the goals stated in the first section? 

\printbibliography

\end{document}
