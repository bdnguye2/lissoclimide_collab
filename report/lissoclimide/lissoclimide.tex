%%%%%%%%%%%%%%%%%%%%%%%%%%%%%%%%%%%%%%%%%%%%%%%%%%%%%%%%%%%%%%%%%%%%%%%%
% Preamble
%%%%%%%%%%%%%%%%%%%%%%%%%%%%%%%%%%%%%%%%%%%%%%%%%%%%%%%%%%%%%%%%%%%%%%%%
\documentclass[11pt]{article}
%
% Packages and other includes
% Pagination
\usepackage[letterpaper, margin=1.25in]{geometry}
%
% Fonts
\usepackage[T1]{fontenc} % best for Western European languages
\usepackage{lmodern} % Latin Modern instead of CM
\usepackage{textcomp} % required to get special symbols
%
% Math
\usepackage{amsmath, amssymb}
\usepackage{braket}
%
% Graphics, floats, tables
\usepackage{graphicx, color, float, array}
%
% Hyperlinks
\usepackage{hyperref}
%
% Bibliography
\usepackage[style=numeric, sorting=none, backend=biber]{biblatex}
\addbibresource{references.bib}
%
% Revision (see Makefile)
%\input{revision.tex}
%
% Definitions and settings
% Paragraph indent and spacing
\setlength{\parskip}{0.4\baselineskip}
\setlength{\parindent}{0in}
%
% Math mode version of "r" column type (requires array package)
\newcolumntype{R}{>{$}r<{$}}
%
% Title, authors, date
\title{\textbf{Research Report}}
\author{YOUR NAME}
\date{MM/DD/YYYY -- MM/DD/YYYY }
%
%
%%%%%%%%%%%%%%%%%%%%%%%%%%%%%%%%%%%%%%%%%%%%%%%%%%%%%%%%%%%%%%%%%%%%%%%%
% Main document
%%%%%%%%%%%%%%%%%%%%%%%%%%%%%%%%%%%%%%%%%%%%%%%%%%%%%%%%%%%%%%%%%%%%%%%%
%

\begin{document}

\maketitle

\section{Introduction}

The first paragraph provides broad context and motivation for your
report. It should establish a connection to your and the group's scientific
long-term goals, and avoid lofty or vague statements. Popularity,
recent public attention, or the observation that ``our group is
interested in A, B and C'' etc. are insufficient motivations. You may
re-use parts of this paragraph from previous reports, taking into
consideration feedback you received. 

The second paragraph also provides context, but much more focused on
your research. For example, you may explain in the first paragraph why
solvation effects are important for chemists, and then give a brief
overview of work on excited state properties in the area of implicit
solvent models. The purpose of this paragraph is to help the reader
understand what prior work was done, and to provide the ``background
scenery'' for the work you are reporting. 

The third paragraph should explain how your work goes beyond existing
methods or results, and define its specific goals and
hypotheses. It is better to focus on what your work aims to do than on
what you and others before you were not able to do. 

While your report should efficiently communicate any recent results you
have had, it also serves to help you gauge your progress, see the
conclusions and broader context of your work, assess priorities,
generate narratives for later use in publications, and sharpen your
writing skills. Even if you spend the entire period of your report
debugging code, your report can help re-state the problem, discuss
strategies you employed to find the error and whether your time was well
used. You might also conclude, e.g., that you need additional resources
and help to tackle the problem, or that it may not be worth pursuing
further. 

The entire report should be written in scientific English suitable for
publication, see \textcite{Strunk99} for an introduction. Try to be
concise, clear, and articulate at the same time. Bullet points,
unexplained abbreviations, incomplete sentences, or slang are not
acceptable. References should be cited using the group's refbase cite
keys. Ensure that all symbols are defined, and figures and tables
are clear and have self-contained captions. For instructions on how to
best display quantitative information in figures, refer to
\textcite{Tufte01}. Use 
vector graphics (eps or pdf) when possible and appropriate, otherwise
600 dpi bitmap (png, jpeg). The report should be between 2 and 4 pages
long, including references.

If you are working on more than one different projects simultaneously,
you may alternate your biweekly reports over a longer period of
time.

In addition to attaching an electronic version of your report to your
meeting agenda, please print it (two-sided) and submit a paper copy to
Filipp's office by 5:00 pm on the day before your meeting (you may slide
it under the door if necessary). 

\section{Methods}

The methods section should concisely state the methods you used to
obtain your results. For a theoretical project, this section will
review key assumptions and equations; for an implementation-centered
project, the main working equations and implementation strategies will
be stated here; for a computational project, this section will contain
the computational details. Depending on the subject, parts of this
section may be carried over from previous reports.

\section{Results}

This section states the main new results of the current
reporting period. These results could be presented, e.g., in the form of
equations, figures, tables, and text. Negative results are just as
important as positive ones and should be carefully reported. Use a
balanced tone avoiding over- and understatement. 

Sometimes, it is helpful to separate statement of the results and
discussion, although a strict separation of the two is often
impossible. A separate discussion section may be appropriate if you are
proposing a speculative rationale or if several interpretations of your
results are possible. If you have no new results at all, this is the
place to explain why. 

\section{Conclusions}


This section should answer the following questions:
What is the significance of your results for the question, problem, or
hypothesis posed in the introduction? What, if any, are the
implications in the broader context of the research area laid out in the
second paragraph of the introduction? 

A second paragraph should discuss any conclusions for your future work. 
Is there a need to change your approach or your priorities? What remains
to be done to reach the goals stated in the first section? 

\section{Service}

This section should summarize group service activities you have engaged
in during the reporting period, including, e.g., related to shared
governance (deputy positions), work for the high school outreach
program, helping or supervising other group members to larger extent
than usual.

\printbibliography

\end{document}
